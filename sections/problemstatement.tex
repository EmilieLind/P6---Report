\chapter{Problem Statement}\label{ch:problemstatement}
As a visitor in a city with which they are unfamiliar, tourists may benefit from local guided tours. As stated by Meged (2010), guided tours are a frequently used way to experience a destination city, and there is no sign that they are in decline \cite{Meged2010}. Because of this, local tourist bureaus, as well as the tourists themselves, benefit when the full potential of a guided tour is utilised. It is therefore interesting to explore new ways to enhance guided tours. 

Besides that, the motivation to travel has changed over time. While the motivation to travel was previously largely based on a desire to escape daily routines, tourists today use their travels not only as a source of entertainment, but also as a way to broaden their horizons and learn something new, according to Nam et al. (2006) \cite{Nam2006}. One thing to keep in mind is that while some people have no problem learning through verbal communication, others' learning is better enabled through visual media. As addressed by Beran et al. (2012), some students learn better through multimedia explanations than through purely verbal ones \cite{Beran2012}. As guided tours must move from one location to another, the possibilities for multimedia applications are limited. Tour goers may, however, bring their personal smartphones, allowing for applications making use of Augmented Reality (AR).

AR is a relatively new technology, and most research regarding it is still recent, leaving a potential for research gaps. What makes AR unique is that the user may experience the real world combined with the digital, as opposed to Virtual Reality (VR), where users mostly focus their senses on the virtual world rather than the real one. Therefore, an AR application may be especially relevant for guided tours as it can both provide additional information and give a realistic sense to the experience, according to Nam et al. (2006) \cite{Nam2006}. This can be utilised to show the tourists scenes they would not otherwise be able to see.

Although research on Augmented Reality exists, there is currently little research done investigating the effects of augmented reality on guided tours, leaving a research gap for us to investigate.

For the purposes of this project, a collaboration has been established with VisitAalborg, who facilitated a collaboration with Inge Vestergaard, a guide from Aalborg Guideforening. This enabled us to research the possible effects of AR on guided tours in Aalborg. These motivations led to the investigation of the following problem statement:

\begin{displayquote}\textit{How does the use of a relevant AR application to showcase otherwise non-visible scenes during a guided tour impact the tour experience?}\end{displayquote}

In the context of this report a tour experience is based on entertainment value and education value. According to the tour guide Inge Vestergaard\todo{Refer to e-mail in appendix}, the degree of these depend very much on the individual groups and their interests. If it is a historically interested group then the educational weighs more than the entertainment, whereas if it is a more casual group the entertainment value may weigh more than the educational value. The success of the experience depends on the guide’s ability to read the group and see which element weighs more for that particular group. Additionally, the case of the AR application being relevant means it is integrated into the narrative of the guided tour.

To test the problem formulation, the following research questions will be investigated: 

\begin{enumerate}
\item When presented with an AR application with no features other than showing a model, for how long will users generally be willing to direct their attention towards the application?
\item In what way does the addition of an AR application affect tour goers’ curiosity to learn more, if at all?
\item In what way does the addition of an AR application affect whether tour goers find the tour entertaining, if at all?
\item In what way does the addition of an AR application affect whether tour goers are distracted during the tour?
\end{enumerate}

\todo{suggestion for transition here.}

To be able to investigate these questions, first a base of theory and background research must be laid. This will be done in the following chapter. 