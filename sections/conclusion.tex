\chapter{Conclusion}\label{ch:conclusion}
In this project, we set out to investigate how the use of a relevant augmented reality (AR) application to showcase otherwise non-visible scenes during a guided tour could impact the tour experience. To answer this problem, four research questions were formed. These concerned for how long a user would be willing to direct attention towards an AR application, as well as how an AR application would affect the tour goers’ curiosity to learn more, the entertainment value of the tour, and the level of distraction during the tour.

At first, a study was made into AR technology and how it works. This study especially focused on  the different tracking systems within AR technology, so that a decision could be made on which method to utilise in the development of the application. In connection to the study of AR thorough research was done into the various visual cues humans use to perceive depth. This was carried out to get a better understanding of the way humans perceive the world as three dimensional, and how this is applicable to the AR technology. This also created a base for what should be added to a model in order to enhance humans’ interpretation of it as being three dimensional. From this research it was found that textures and shaders are considered of high importance when it comes to cueing humans about objects as belonging to the 3D world. Lastly, a study was made into the state of the art of AR solutions which aim towards tourists. This was performed in order to look into the existing systems which aim toward similar goals as this application would try to accommodate.

The first research question deals with how long participants will look at an AR application with only a model and no other features. To investigate this a test was conducted with an application that fit this criterion, since we had not yet developed our own application. The average time was found to be between 30 and 50 seconds, with no outside factors affecting the experience.

In order to assess the problem statement, an AR application was developed for the purpose of conducting an experiment during a mock-up guided tour at Rakkerens Hule, an underground dungeon which is rarely accessible. Using AR, this application allows users to view a virtual replication of the dungeon which can be projected onto a physical marker on the ground using their smartphone’s camera.

The experiment was conducted with the help of a tour guide who facilitated the mock-up tour. Two groups of volunteers participated in the experiment, one of which used the AR application to see Rakkerens Hule during the tour --- the other group served as a control group. During the experiment, data was gathered through video, questionnaires, and a semi-structured interview with the tour guide.

The results from the evaluation showed that there was no apparent difference between having the AR incorporated in the guided tour compared to not having it. However, a larger sample size together with implementing AR in other parts of the tour could influence the results. Further tests need to be conducted to draw a more definite conclusion.