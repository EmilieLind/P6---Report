\chapter{Conclusion}\label{ch:conclusion}
In this project, we set out to investigate how the use of a relevant augmented reality (AR) application to showcase otherwise non-visible scenes during a guided tour could impact the tour experience. To answer this problem, four research questions were formed. These concerned for how long a user would be willing to direct attention towards an AR application, as well as how an AR application would affect the tour goers’ curiosity to learn more, the entertainment value of the tour, and the level of distraction during the tour.

First, a study was made into the various visual cues humans use to perceive depth. This was carried out to get a better understanding of the way humans perceive the world as three dimensional, and how this is applicable to the AR technology. This also created a base for what should be added to a model in order to enhance humans’ interpretation of it as being three dimensional. From this research it was found that textures and shaders are considered of high importance when it comes to cueing humans about objects as belonging to the 3D world. Next, a study was made into AR technology and how it works. This study especially focused on the different tracking systems within AR technology, so that a decision could be made on which method to utilise in the development of the application.  Lastly, a study was made into the state of the art of AR solutions which aim towards tourists. This was performed in order to look into the existing systems which aim toward similar goals as this application would try to accommodate.

The first research question deals with how long participants will look at an AR application with only a model and no other features. To investigate this a test was conducted with an application that fit this criterion, since we had not yet developed our own application. The average time was found to be between 30 and 50 seconds, with no outside factors affecting the experience.

In order to assess the problem statement, an AR application was developed for the purpose of conducting an experiment during a mock-up guided tour at Rakkerens Hule, an underground dungeon which is rarely accessible. Using AR, this application allows users to view a virtual replication of the dungeon which can be projected onto a physical marker on the ground using their smartphone’s camera.

The experiment was conducted with the help of a tour guide who facilitated the mock-up tour. Two groups of volunteers participated in the experiment, one of which used the AR application to see Rakkerens Hule during the tour---the other group served as a control group. During the experiment, data was gathered through video, questionnaires, and a semi-structured interview with the tour guide.

The results from the questionnaires showed no significant difference between the two groups in terms of curiosity, entertainment value, or distraction. However, several participants of the test group expressed an interest in the virtual representation, and some from the control group expressed a desire to see a visual representation of Rakkerens Hule. The guide herself expressed that she felt no significant difference in the tours whether AR was implemented or not, but found the concept interesting, and expressed an interest in further development. A video analysis showed that the guide had a slight tendency to use more pointing gestures when tour goers were using the AR application, as well as gestures which mimicked the shape of objects within the dungeon. This may be because she knows that the tour goers have a reference point from which to imagine the dungeon.

In the control group, there were 50 cases of participants looking away from the guide and tour site---that is, an average of 4.54 cases per participant. In the test group, there were 98 cases, resulting in an average of 7.54 cases per participant. Of these 98 cases, 43 came from two of the 13 participants, and these can be considered outliers. If these two outliers are discounted, the test group had 55 cases, for an average of 5 cases per participant. Thus, the difference in distractions between the two groups is not large. Test participants generally used the AR application after the guide was finished talking about the site in question. This may have been out of respect for the guide, so as to not disturb the narrative. It may also be due to participants being absorbed in the narrative. 

Based on the video analysis and the questionnaire results, it can be concluded that the addition of an AR application does not obstruct the guided tour experience. Further testing is needed to draw a definite conclusion on which positive effects AR may have on a guided tour. A larger sample size is recommended, as well as a tour which implements AR in several sites rather than just the one. However, as the AR application showed no signs of obstructing the guided tour, it can be concluded that it has potential to be integrated into such tours, to provide tour goers with a visual representation of a site. This could be a site which is not available, or which no longer resembles the original site.