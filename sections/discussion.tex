\chapter{Discussion}\label{ch:discussion}
Throughout the project, there were some aspects that could have been done in a different manner, and, in that way, the overall quality of the project could have been improved. In this chapter, those aspects are described in detail.

\section{Quality of results}
The data samples collected from the test have been very small. Therefore, making general conclusions based on standalone data would be considered very unreliable in this case. For instance, there were 11 and 13 participants in the control group and test group respectively. Meanwhile, there were seven rating options on the rating scale for the participants to select from in the questionnaire. Therefore, if the participants differed much in their opinion about the experience, the result could easily be so spread out that the data collected would be meaningless. In relation to this concern is another factor that may have reduced the reliability of the test which is the limitation of time for the test (one hour), since this put a restriction to how many groups could experience the tour. Because that, for each of the two conditions there would be only one group testing which have made it impossible to seek for patterns to generalise from e.g. by averaging numbers.

Since the marker used in the evaluation had a white background, there was some trouble regarding its' detection. This happened because white color is good at reflecting light, which made the marker appear very bright on a sunny day. Therefore, the detection of features was a more harder task to accomplish. Those troubles
with detection might have confused some of the participants who were having the AR guided tour. For that reason, the overall impression from the AR guided tour might have been lessened. 

In the questionnaire, there may be some disruptive data connected to the SAM based question concerning the participants’ arousal, cf. Appendix \ref{app:questionnaire}. This was due to poor understanding of the word among the participants in this particular context, which can have had an impact on the reliability of this specific result.

Concerning the results received from the video analyses the video data collected has been inflicted by a series of external disruptions like loud noise and high traffic with moving vehicles --- known conditions, though beyond the group’s control. However, this could not be avoided due to the changing nature of a tour site. Doing the test at a quieter place of course would be a possibility, but then the test would suffer in the feel of actual being on the site watching what lies underground. Therefore, although the number of distractions among the participants observed in the videos may have increased due to the noise and traffic in the background, it is considered to be the most realistic scenario and therefore desirable. Furthermore, these conditions may also have influenced the answers given in the questionnaire. Especially in regards to answers given to the questions 13 and 14 asking into the participants’ experience of distraction at the site, cf. Appendix \ref{app:questionnaire}.
Both for the video data and the questionnaire answers it is therefore reasonable to consider parts of the data to be unreliable. 

To heighten the validity of the results, different forms of data collection methods (questionnaire, video recording, and interview) have been used to gather both quantitative and qualitative data, which help balance each other out if used in combination . Furthermore, this has been done to enable the usage of methodological triangulation also referred to as mixed method research \cite{Kennedy}. Using triangulation can validate the results by cross verification as it investigates whether or not the varying forms of data confirm each other, and thereby reaches to the same conclusion. Moreover, the triangulation of data helps to rectify bias. This could, for instance, be procedural bias such as participants who rush their answers in order to quickly finish of the questionnaire, or measurement bias like response bias, which can be counteracted by combining self-reported and observational research \cite{Kennedy}. Although the latter could have been improved by using investigative triangulation for the analysis of the videos, since two researchers are likely to have different perspectives. Therefore a utilisation of two or more observers that counterbalance one another would improve the validity of the results when compared to only having one investigator perform the analysis \cite{Kennedy}. The lack of data logged from the application is a deficiency in the gathered data. This could have illustrated how participants used the application e.g. activation time and orientation, which could have added to the quantitative data collection. The reason for leaving out data logging was due to the Danish legislation about privacy policy concerned with using data collected from people’s phones cf. The Danish Data Protection Agency \cite{Datatilsynet}.

A factor to consider in the execution of the test that might have affected the validity is the participants’ backgrounds. Although the guide explained that the people going on tours come from a variety of backgrounds such as families with parents and children, elderly people, bachelor parties, etc., cf. Appendix \ref{app:guide_emails}, the participants used for the test consisted only of people within the age range 20 to 31 and were recruited primarily among students. This has resulted in the sample having an omission bias, for which reason it cannot be said to be truly representative of the users typically using the guided tours. This could easily have been managed differently by a change in the procedure for recruiting participants. This was done by posting an event on Facebook. It was shared only within the medialogy group and among the group members’ friends and families together with recruiting locally by asking medialogy students at campus.

Another factor which could influence the validity of the test is the recruited participants’ self-interest, both in regard to their personal interest to the content of the tour as well as their interest in the AR technology. As mentioned above, many of the recruited participants were medialogy students who presumably have a technical background with an interest in new technologies. This could potentially make them biased towards a higher ranking of the tour version with AR incorporated. Although the questionnaire results imply this to not be the case, this could be due to the procedure not having each of the two groups try out both versions. For this reason, the participants’ were not able to compare the two versions. Choosing this procedure therefore may have increased the validity in regards to this aspect.  

\section{Future work}
There are also several aspects of the project that could be interesting to look into as part of future research. 

One of them is to test a VR version of the design in order to acquire the knowledge about how VR would influence the guided tour experience compared to AR. This might be interesting to compare based on it being a lot easier to leave out the AR aspect of an app and just have an entirely digitally rendered version of the app. This would allow tour goers to also look at the model later without being in the correct spot. However, it might lessen the immersion of the tour that the thing being shown on screen has no direct link to the surroundings. This makes it an interesting angle to research, since it would be simpler to make, but might lose some immersion compared to AR. 

Another aspect that would be curious to investigate is a markerless AR implementation of the product. This would allow for an easier use of the final product since there would not be a necessity to carry around a rather large marker during the guided tour.
  
Another possible point of interest arises due to how the human cognition works. To be more exact, humans process auditory and visual information through different channels, and, unfortunately, there is a possibility to overload those channels. In order to increase the effectiveness of acquiring new information, an individual should be able to mentally connect the visual and auditory information and organise it in a coherent manner \cite{audiovisual_learning}. Therefore, it would be interesting to see how to make the final product in such a way that the aforementioned possible complications could be avoided and the overall user experience could be improved.

Implementing realistic lighting could also be a part of future work, since that could make the 3D model look more natural in the outdoor setting. In that way, the 3D model would seem more like a part of the urban environment, and less like a foreign element. Thus, the overall user experience could be improved.

Furthermore, another possibility to improve the design could be to integrate actual pictures of the site into an AR system, as suggested by Inge Vestergaard, who is the guide that aided with the evaluation testing. Besides that, integrating the AR on other sites of the tour might be a good idea, since it would give a better understanding to the users of how certain urban elements might have looked in the past, and give results that are more influenced by the application so they will be more clearly comparable. 

The main reason why the aforementioned aspects are not part of the current report is lack of time due to organisational issues. They were mainly caused by having some trouble with establishing a collaboration with a third party. At first, there was made considerable effort to work together with a museum. However, the museum association did not show interest in our project. After the official rejection, VisitAalborg was contacted, which then resulted in a collaboration. 

