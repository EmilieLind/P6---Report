\chapter{Evaluation}\label{ch:evaluation}
\section{Objective and Method} 
Set in relation to the problem formulation, the purpose of this evaluation is to see if using AR during a guided tour will have either a significant positively or negatively effect on the user experience when compared to a regular guided tour without the usage of AR. This is done in order to test whether incorporating AR will be beneficial during guided tours or not. 

To test this two similar versions of a guided tour has been arranged in cooperation with a tour guide from Aalborg Guideforening connected to the project through VisitAalborg - one with AR and one without AR. For the test to not suffer from fatigue by a repetitive experience within the participants testing the two tour versions, the research method between-subjects (often used within the scientific discipline of psychology when comparing two conditions [Citation] \todo{cite{Charness et al. 2011})} is chosen to be used. This method is used together with the quantitative methodology \textit{survey research} - used as the primary source for data collection.

The data collected from the conducted experiment by Likert scales is considered ordinal measurements. To analyze ordinal responses a nonparametric test is recommended to be used. Moreover the design of the experiment is unpaired, meaning each participant only tests one of the two tour versions. Because of this the nonparametric Wilcoxon Rank Sum Test should be used for the analysis. For the calculations of the data the significance level has been set to 0.05.

\todo{Missing: Approach inspired by add sources, References to similar studies and their test methods, Experience aspects/parameter}

\section{Setting}
The test will consist of a guided tour of app. 30 minutes, with two groups of participants. It will take place in Aalborg city centre during the morning. The guide will be Inge Vestergaard from Aalborg Guideforening. The guided tour will start at Gammeltorv then continue onto the area where Rakkerens Hule is located. Here one group will experience the AR application. The tour will finish off with the old monastery Helligåndsklosteret before returning to Gammeltorv. 

During the test two group members will follow the tour with video cameras and the necessary marker for the AR application. The other two group members will stay at Gammeltorv to prepare the second group and to be ready to hand out the end of test questionnaires to the first group. 

This makes the required items needed for this test:
\begin{itemize}
\item A video camera
\item A marker for the AR application
\item The questionnaires
\item Consent forms
\item The participants' android phones
\item Clipboards for the questionnaires
\end{itemize}



\section{Procedure}
\todo{Describe how we got the participants}
For the test a timeline was created, since it was only being conducted once and it required coordinating with both the guide, the project group and the participants. Beforehand the participants were split into two groups so as the second group would not have to wait for the first group to finish. 

\begin{tabular}{l p{12cm}}
09:00 & - The project group meets up at Gammeltorv to prepare \\
09:45 & - Evaluation team 0 meet up at Gammeltorv \\
 & - Consent forms are presented and the procedure is explained \\
10:00 &   - Evaluation Team 0 starts the guided tour \\
10:10 & - Evaluation Team 1 meets up at Gammeltorv \\
 & - Consent forms are presented and the procedure is explained including the AR application \\
10:30 & - Evaluation Team 0 finishes the guided tour and are handed the questionnaires \\
 & - Evaluation Team 1 starts the guided tour
\\ 
11:00 & - Evaluation Team 1 finishes the guided tour and are handed the questionnaires \\
11:xx & - When Evaluation Team 1 is finished the test ends \\
\end{tabular}


\section{Statistical analysis of results} 
\subsection{hypothesis}
Hypothesis based on research question 2:

In what way does the addition of an AR application affect tour goers’ curiosity to learn more, if at all?

\begin{itemize}
\item H1(0) - After experiencing a tour with the AR application, participants report the same level of curiosity as that of the participants who experienced the tour without the AR application. 
\item HA - After experiencing a tour with the AR application, participants report a different level of curiosity than that of the participants who experienced the tour without the AR application. 
\begin{itemize}
\item H1a: After experiencing a tour with the AR application, participants report a higher level of curiosity than that of the participants who experienced the tour without the AR application.
\end{itemize}
\end{itemize}

Hypothesis based on research question 3:

In what way does the addition of an AR application affect whether tour goers find the tour entertaining, if at all?

\begin{itemize}
\item H2 - After experiencing a tour with the AR application, participants agree with the statement that they were entertained to the same degree as that reported by the participants who experienced the tour without the AR application. 
\item HA - After experiencing a tour with the AR application, participants agree with the statement that they were entertained to a different degree than that reported by the participants who experienced the tour without the AR application. 
\begin{itemize}
\item H2a: After experiencing a tour with the AR application, participants agree with the statement that they were entertained to a higher degree than that reported by the participants who experiences the tour without the AR application. 
\end{itemize}
\end{itemize}

Hypothesis based on research question 4:

In what way does the addition of an AR application affect whether tour goers are distracted during the tour?

\begin{itemize}
\item H3 - After experiencing a tour with the AR application, participants report the same level of distraction as that of the participants who experienced the tour without the AR application. 
\item HA - After experiencing a tour with the AR application, participants report a different level of distraction than that of the participants who experienced the tour without the AR application.
\begin{itemize}
\item H3a: After experiencing a tour with the AR application, participants report a higher level of distraction than that of the participants who experienced the tour without the AR application.
\end{itemize}
\end{itemize}