\pdfbookmark[0]{English title page}{label:titlepage_en}
\aautitlepage{%
  \englishprojectinfo{
    Augmented Reality Tour %title
  }{%
    Interactive Systems Design %theme
  }{%
    Spring Semester 2017 %project period
  }{%
    MTA17630 % project group
  }{%
    %list of group members
    Christina Kristensen\\ 
    Emilie Lind Damkjær\\
    Liv Arleth\\
    Margarita Kaljuvee
  }{%
    %list of supervisors
    Ivan Adriyanov Nikolov\\
    Claus Brøndgaard Madsen
  }{%
    1 % number of printed copies
  }{%
    \today % date of completion
  }%
}{%department and address
  \textbf{Medialogy}\\
  Aalborg University\\
  \href{http://www.aau.dk}{http://www.aau.dk}
}{% the abstract
This is a research study into how augmented reality, when used to showcase a construction normally inaccessible to the audience, may impact the audience’s experience with a guided tour concerning curiosity invocation, entertainment value, and distraction level. An AR application was developed, showing a 3D model of a subterranean structure which becomes visible using the camera of a smartphone and a marker. To research this the method \textit{between design} was applied to a mock-up tour arranged in cooperation with a guide from Aalborg Guideforening. In accordance with this method, a control group would experience the tour through the guide’s normal narratives, and a test group would use the application in addition to the tour. By comparing results gathered from the two groups, a conclusion is drawn that the AR application does not obstruct the tour experience, and that it therefore has the potential to be integrated in an actual guided tour after further development and testing.}